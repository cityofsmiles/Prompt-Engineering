\documentclass[14pt]{beamer}
\usetheme{default}
\usecolortheme{dove} % Clean, white background
\setbeamertemplate{navigation symbols}{} % Remove clutter

% High-impact font settings
\usepackage{helvet}
\renewcommand{\familydefault}{\sfdefault}

\begin{document}

% --- SLIDE 1: The Hook ---
\begin{frame}
    \centering
    \Huge \textbf{AI in the Math Class}
\end{frame}

% --- SLIDE 2: The Fear ---
\begin{frame}
    \centering
    \Huge \textit{"It just gives the answer."}
\end{frame}

% --- SLIDE 3: The Reality ---
\begin{frame}
    \centering
    \Huge \textbf{Wrong.}
\end{frame}

% --- SLIDE 4: The Analogy ---
\begin{frame}
    \centering
    \Large It’s not a calculator.\\
    \vspace{1cm}
    \Huge It’s a \textbf{Tutor.}
\end{frame}

% --- SLIDE 5: Prompt Engineering ---
\begin{frame}
    \centering
    \Huge \textbf{Prompt Engineering}
\end{frame}

% --- SLIDE 6: Definition ---
\begin{frame}
    \centering
    \Large The art of \textbf{Context}.
\end{frame}

% --- SLIDE 7: Example Problem ---
\begin{frame}
    \centering
    \Large Solve for $x$: \\
    \vspace{0.5cm}
    $2x^2 - 4x - 6 = 0$
\end{frame}

% --- SLIDE 8: Bad Prompt ---
\begin{frame}
    \centering
    \small \texttt{"Solve this equation."}
\end{frame}

% --- SLIDE 9: Bad Output ---
\begin{frame}
    \centering
    \Huge \textbf{X = 3, -1} \\
    \large (No explanation)
\end{frame}

% --- SLIDE 10: The Lessig "Echo" ---
\begin{frame}
    \centering
    \Huge \textbf{Better Context}
\end{frame}

% --- SLIDE 11: Role Prompting ---
\begin{frame}
    \centering
    \Large "Act as a \textbf{Socratic Tutor}."
\end{frame}

% --- SECTION: ROLE PROMPTING ---
\begin{frame}
    \centering
    \Huge \textbf{1. Role Prompting}
\end{frame}

\begin{frame}
    \centering
    \Large "You are an \textbf{expert examiner}."
\end{frame}

\begin{frame}
    \centering
    \Large Create a rubric for a \\ \textbf{Calculus} project.
\end{frame}

% --- SECTION: FEW-SHOT LEARNING ---
\begin{frame}
    \centering
    \Huge \textbf{2. Few-Shot Learning}
\end{frame}

\begin{frame}
    \centering
    \Large Give the AI \textbf{Examples}.
\end{frame}

\begin{frame}
    \centering
    \small 
    Q: Solve $3x = 12$. A: $x=4$ \\
    \vspace{0.3cm}
    Q: Solve $5x = 25$. A: $x=5$ \\
    \vspace{0.3cm}
    \Huge \textbf{Model the logic.}
\end{frame}

% --- SECTION: CHAIN OF THOUGHT ---
\begin{frame}
    \centering
    \Huge \textbf{3. Chain-of-Thought}
\end{frame}

\begin{frame}
    \centering
    \Large "Think \textbf{step-by-step}."
\end{frame}

\begin{frame}
    \centering
    \Large Force the LLM to verify \\ \textbf{intermediate} steps.
\end{frame}

\begin{frame}
    \centering
    \Large Avoid the "Hallucination" of \\ \textbf{Arithmetic Errors}.
\end{frame}

% --- SLIDE 12: Chain of Thought ---
\begin{frame}
    \centering
    \Large "Explain the \textbf{Quadratic Formula} step-by-step."
\end{frame}

% --- SLIDE 13: Mathematical Formatting ---
\begin{frame}
    \centering
    \Large Use \textbf{LaTeX} output only.
\end{frame}

% --- SLIDE 14: The Formula ---
\begin{frame}
    \centering
    $$x = \frac{-b \pm \sqrt{b^2 - 4ac}}{2a}$$
\end{frame}

% --- SLIDE 15: Constraint ---
\begin{frame}
    \centering
    \Large "Don't give the answer... \\ \textbf{guide} the student."
\end{frame}

% --- SLIDE 16: Closing ---
\begin{frame}
    \centering
    \Huge \textbf{Empower} your students.
\end{frame}

\end{document}
